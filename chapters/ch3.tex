%% This section adds chapter 3
\section{Research methods}

This thesis delves into the complex motivations behind developer participation in open-source projects by combining a \ac{slr} and an in-depth case study. The literature review establishes a robust foundation of knowledge, while the case study illuminates the nuanced social interactions and experiential factors shaping individual developers' decisions to contribute.

\subsection{Systematic Literature Review} \label{slr}

\ac{slr} offers a disciplined research method that promotes a clear, targeted investigation through a well-defined question. This study employs the \ac{slr} methodology as outlined by Kitchenham, ensuring a reliable and unbiased review \citep{Kitchenham}.

\subsubsection{Systematic Literature Review definition}

\ac{slr} is a meticulously planned research approach aiming to identify, evaluate, and synthesize all available evidence related to a clearly defined research question \citep{Kitchenham}. It employs a transparent, reproducible protocol to minimize bias, emphasizing comprehensiveness and critical appraisal of included studies.

The methodology consists of three main phases. First, in the planning phase, researchers must clearly define the research questions that guide the entire review. A detailed review protocol should then be developed, outlining the search strategy (databases, search terms), study selection criteria (e.g., publication types and dates), quality assessment methods, data extraction plans, and the strategy for synthesizing the findings.

The second phase involves conducting the review. This includes identifying research using the search strategy, selecting studies based on the eligibility criteria, critically evaluating study quality, extracting the relevant data, and finally, synthesizing that data using techniques like meta-analysis or thematic analysis.

Finally, the reporting phase focuses on transparently documenting the entire SLR process for reproducibility and critical assessment. Researchers must present the results in a clear manner, addressing the initial research questions and their implications, while also acknowledging any limitations within the review process.

\subsubsection{Systematic Literature Review approach}

Open source software development presents a unique model of collaboration where developers voluntarily contribute their time and expertise. Unlike traditional software engineering environments, motivations in open source extend beyond direct financial compensation. Researchers have investigated these motivations from various perspectives, including psychological, economic, and social factors. This complex landscape can lead to potentially varied interpretations of what drives participation.

The SLR will lay the groundwork for my thesis by providing a thorough understanding of current research on open source developer motivations. Based on the SLR findings, I will be able to identify under-investigated areas or potential gaps in the theoretical frameworks used to explain developer behavior.

% \subsection{Case study research}

% The concept of the "case study" originated within software engineering literature during the late 1970s \citep{runeson2012case}, providing a formalized approach to qualitative research within the field. Guided by the framework by Runeson \citep{runeson2012case}, this thesis will utilize in-depth developer interviews to investigate motivations behind open source project contributions.

% \subsubsection{Case study definition}

% A case study is an in-depth research method that involves the detailed investigation of a specific subject (individual, group, organization, event, or phenomenon) within its real-world context \citep{yin2003case}. 

% Runeson and Höst propose a framework to ensure the rigor and focus of case study research  \citep{runeson2012case}. The framework begins with clear research questions that evolve throughout the study.  These open-ended questions guide the exploration of the chosen subject. Propositions act as hypotheses or theoretical assumptions about the case, providing a lens for data interpretation. Defining the units of analysis, such as individuals, projects, or processes, is crucial for setting the boundaries of the study. The framework emphasizes the need to outline the logic connecting collected data to the propositions, including anticipated analytical techniques. Finally, it's essential to establish criteria for interpreting the findings and use triangulation with multiple data sources to ensure the study's validity.

% \subsubsection{Case study approach}

% While this thesis investigates various aspects of open-source project motivation, it will employ a case study approach to address the specific research questions focused on barriers and challenges faced by developers. This in-depth qualitative method is ideal for exploring the complexities of these experiences and the strategies developers use to navigate them.  The research will examine individual developer accounts to understand the nature of these challenges, which might encompass technical hurdles, community dynamics, time constraints, or other factors.

% The case study approach will facilitate a nuanced exploration of how developers overcome or adapt to these obstacles. Probing questions will uncover successful strategies, potential coping mechanisms, and the resources developers leverage to maintain their participation in open-source projects. By analyzing these first-hand accounts, this thesis aims to identify common themes related to challenges and resilience, offering valuable insights to support and sustain open-source communities.

\clearpage  % This command will start the next section from the new page


