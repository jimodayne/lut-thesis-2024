%% This section adds chapter 3
\section{Research methods}

This thesis delves into the complex motivations behind developer participation in open-source projects by combining a \ac{slr} and an in-depth case study. The literature review establishes a robust foundation of knowledge, while the case study illuminates the nuanced social interactions and experiential factors shaping individual developers' decisions to contribute.

\subsection{Systematic Literature Review} \label{slr}

\ac{slr} offers a disciplined research method that promotes a clear, targeted investigation through a well-defined question. This study employs the \ac{slr} methodology as outlined by \citet{Kitchenham}, ensuring a reliable and unbiased review.

\subsubsection{Systematic Literature Review definition}

\ac{slr} is a meticulously planned research approach aiming to identify, evaluate, and synthesize all available evidence related to a clearly defined research question \citep{Kitchenham}. It employs a transparent, reproducible protocol to minimize bias, emphasizing comprehensiveness and critical appraisal of included studies.

The methodology consists of three main phases. First, in the planning phase, researchers must clearly define the research questions that guide the entire review. A detailed review protocol should then be developed, outlining the search strategy (databases, search terms), study selection criteria (e.g., publication types and dates), quality assessment methods, data extraction plans, and the strategy for synthesizing the findings.

The second phase involves conducting the review. This includes identifying research using the search strategy, selecting studies based on the eligibility criteria, critically evaluating study quality, extracting the relevant data, and finally, synthesizing that data using techniques like meta-analysis or thematic analysis.

Finally, the reporting phase focuses on transparently documenting the entire SLR process for reproducibility and critical assessment. Researchers must present the results in a clear manner, addressing the initial research questions and their implications, while also acknowledging any limitations within the review process.

\subsubsection{Systematic Literature Review approach}

Open source software development presents a unique model of collaboration where developers voluntarily contribute their time and expertise. Unlike traditional software engineering environments, motivations in open source extend beyond direct financial compensation. Researchers have investigated these motivations from various perspectives, including psychological, economic, and social factors. This complex landscape can lead to potentially varied interpretations of what drives participation.

The SLR will lay the groundwork for my thesis by providing a thorough understanding of current research on open source developer motivations. Based on the SLR findings, I will be able to identify under-investigated areas or potential gaps in the theoretical frameworks used to explain developer behavior.


\clearpage  % This command will start the next section from the new page


