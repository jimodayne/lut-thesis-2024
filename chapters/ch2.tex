%% This section adds chapter 2
\section{Research questions} \label{researchQuestions}
This thesis seeks to explain the factors influencing developer participation in open-source projects. To achieve this, the study will delve into three key areas. First, it will explore the primary motivations driving developers to contribute their time and expertise to collaborative open-source endeavors.  Second, the research will investigate the impact of social dynamics within these communities. This includes examining how interactions with other developers and potential networking opportunities influence a developer's decision to participate and their ongoing engagement. I would like to approach these two questions using a Systematic Literature Review methodology, which will be discussed in detail in chapter \ref{slr}.

Finally, the study will identify the barriers and challenges that developers encounter when working on open-source projects. It will then explore the strategies and solutions employed by developers to navigate these obstacles and ensure continued participation. By addressing these multifaceted aspects, this thesis aims to provide a comprehensive understanding of the dynamics shaping developer involvement in the open-source landscape.


\subsection{What are the primary motivations driving developers to participate in open-source software projects?}

This question seeks to understand the underlying reasons why developers choose to contribute their time, skills, and expertise to collaborative software development efforts that are open and freely accessible to the public. By exploring this question, the study aims to uncover the diverse range of factors that incentivize developers to engage in open-source projects. These motivations may vary widely among individuals and can include a combination of intrinsic and extrinsic factors.

\subsection{To what extent do social dynamics, such as community interactions and networking opportunities, impact developer participation in open-source software projects?}

Open-source software thrives on the contributions of volunteer developers. While individual motivations to participate have been explored, a gap exists in understanding how the social environment itself fosters engagement. The second question investigates the impact of social dynamics within open-source projects. Specifically, it examines how interactions within the community and opportunities to build professional networks influence the level of developer participation. By studying these dynamics, the study aims to illuminate how open-source communities can be nurtured to maximize developer engagement and project success.


\subsection{What barriers or challenges do developers encounter when participating in open-source software projects?}

Last, a crucial aspect of this research involves understanding the roadblocks developers encounter when contributing to open-source projects. This includes technical hurdles like complex codebases, time constraints, and communication challenges within geographically dispersed communities. Additionally, factors like unclear project direction and unfamiliarity with open-source etiquette can hinder participation. By identifying these barriers, the study aims to provide insights into how developers navigate these challenges and continue to contribute to open-source projects.


\clearpage  % This command will start the next section from the new page
