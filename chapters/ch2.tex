%% This section adds chapter 2
\section{Research questions} \label{researchQuestions}
This thesis seeks to explain the factors influencing developer participation in open-source projects. To achieve this, the study will delve into three key areas. It will first examine the main reasons behind developers contributing their time and skills to cooperative open-source projects. Second, the research will investigate the impact of social dynamics within these communities. This includes examining how interactions with other developers and potential networking opportunities influence a developer's decision to participate and their ongoing engagement. I would like to approach these two questions using a Systematic Literature Review methodology, which will be discussed in detail in chapter \ref{slr}.

In the end, the research will pinpoint the obstacles and difficulties that developers face while collaborating on open-source projects. This thesis attempts to offer a thorough grasp of the dynamics influencing developer engagement in the open-source scene by tackling these complex elements.


\subsection{What are the primary motivations driving developers to participate in open-source software projects?}

The purpose of this inquiry is to learn more about the fundamental motivations behind developers' decisions to give their time, abilities, and knowledge to open, publicly available collaborative software development projects. The study intends to identify the wide variety of characteristics that encourage developers to work on open-source projects by investigating this subject. These incentives might come from a mix of internal and external sources and can differ greatly amongst people.


\subsection{To what extent do social dynamics, such as community interactions and networking opportunities, impact developer participation in open-source software projects?}

Open-source software thrives on the contributions of volunteer developers. While individual motivations to participate have been explored, a gap exists in understanding how the social environment itself fosters engagement. The second question investigates the impact of social dynamics within open-source projects. Specifically, it examines how interactions within the community and opportunities to build professional networks influence the level of developer participation. By studying these dynamics, the study aims to illuminate how open-source communities can be nurtured to maximize developer engagement and project success.


\subsection{What barriers or challenges do developers encounter when participating in open-source software projects?}

Last, a crucial aspect of this research involves understanding the roadblocks developers encounter when contributing to open-source projects. This includes technical hurdles like complex codebases, time constraints, and communication challenges within geographically dispersed communities. Additionally, factors like unclear project direction and unfamiliarity with open-source etiquette can hinder participation. The challenges faced by developers can greatly influence their willingness to participate in open-source projects. To create a more welcoming and inclusive community, these obstacles must be tackled head-on.


\clearpage  % This command will start the next section from the new page
